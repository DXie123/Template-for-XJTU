\documentclass[UTF8]{XJTUthesis}

\begin{document}

\firstpage{基于深度学习的英飞凌SOT323/343芯片}{质量缺陷检测智能化图像识别算法研究}{电气}{ba}{吧}{ba}{吧}{ba}{吧}

\tableofcontents

\begin{abstract}
而我一直认为,人工智能正是这个时代的范式,而我本人也对这方面颇有兴趣,在本文中,我将结合我自己对人工智能的理解,从历史,现在和未来三个角度对其进行分析。\par
老师的讲课方式是我一直以来所认同和期待的,很多历史上的伟大发明和创造,老师都能结合当时的时代情况,从一个不同的视角做出分析,得出其伟大的原因。正如中国的历史学家说,“以史为鉴,可知兴替”,从我们现在的视角去看过去的发明和创造,更加能指导我们现在的发明和创造。这其实也和人工智能的学习方式所类似,我们以历史上这些推动社会发展的创造为训练数据,让机器进行学习, 再以现在的社会现状等各方面因素作为输入,便可得到在未来,什么会是又一个改变世界的创举。
\end{abstract}
\keywords{未来;未来;未来;未来}

\section{前言}
老师的讲课方式是我一直以来所认同和期待的,很多历史上的伟大发明和创造,老师都能结合当时的时代情况,从一个不同的视角做出分析,得出其伟大的原因。正如中国的历史学家说,“以史为鉴,可知兴替”,从我们现在的视角去看过去的发明和创造,更加能指导我们现在的发明和创造。这其实也和人工智能的学习方式所类似,我们以历史上这些推动社会发展的创造为训练数据,让机器进行学习, 再以现在的社会现状等各方面因素作为输入,便可得到在未来,什么会是又一个改变世界的创举。\par
而我一直认为,人工智能正是这个时代的范式,而我本人也对这方面颇有兴趣,在本文中,我将结合我自己对人工智能的理解,从历史,现在和未来三个角度对其进行分析。\par
老师的讲课方式是我一直以来所认同和期待的,很多历史上的伟大发明和创造,老师都能结合当时的时代情况,从一个不同的视角做出分析,得出其伟大的原因。正如中国的历史学家说,“以史为鉴,可知兴替”,从我们现在的视角去看过去的发明和创造,更加能指导我们现在的发明和创造。这其实也和人工智能的学习方式所类似,我们以历史上这些推动社会发展的创造为训练数据,让机器进行学习, 再以现在的社会现状等各方面因素作为输入,便可得到在未来,什么会是又一个改变世界的创举。\par
而我一直认为,人工智能正是这个时代的范式,而我本人也对这方面颇有兴趣,在本文中,我将结合我自己对人工智能的理解,从历史,现在和未来三个角度对其进行分析。\par
老师的讲课方式是我一直以来所认同和期待的,很多历史上的伟大发明和创造,老师都能结合当时的时代情况,从一个不同的视角做出分析,得出其伟大的原因。正如中国的历史学家说,“以史为鉴,可知兴替”,从我们现在的视角去看过去的发明和创造,更加能指导我们现在的发明和创造。这其实也和人工智能的学习方式所类似,我们以历史上这些推动社会发展的创造为训练数据,让机器进行学习, 再以现在的社会现状等各方面因素作为输入,便可得到在未来,什么会是又一个改变世界的创举。\par
而我一直认为,人工智能正是这个时代的范式,而我本人也对这方面颇有兴趣,在本文中,我将结合我自己对人工智能的理解,从历史,现在和未来三个角度对其进行分析。\par
老师的讲课方式是我一直以来所认同和期待的,很多历史上的伟大发明和创造,老师都能结合当时的时代情况,从一个不同的视角做出分析,得出其伟大的原因。正如中国的历史学家说,“以史为鉴,可知兴替”,从我们现在的视角去看过去的发明和创造,更加能指导我们现在的发明和创造。这其实也和人工智能的学习方式所类似,我们以历史上这些推动社会发展的创造为训练数据,让机器进行学习, 再以现在的社会现状等各方面因素作为输入,便可得到在未来,什么会是又一个改变世界的创举。\par
而我一直认为,人工智能正是这个时代的范式,而我本人也对这方面颇有兴趣,在本文中,我将结合我自己对人工智能的理解,从历史,现在和未来三个角度对其进行分析。\par
老师的讲课方式是我一直以来所认同和期待的,很多历史上的伟大发明和创造,老师都能结合当时的时代情况,从一个不同的视角做出分析,得出其伟大的原因。正如中国的历史学家说,“以史为鉴,可知兴替”,从我们现在的视角去看过去的发明和创造,更加能指导我们现在的发明和创造。这其实也和人工智能的学习方式所类似,我们以历史上这些推动社会发展的创造为训练数据,让机器进行学习, 再以现在的社会现状等各方面因素作为输入,便可得到在未来,什么会是又一个改变世界的创举。\par
而我一直认为,人工智能正是这个时代的范式,而我本人也对这方面颇有兴趣,在本文中,我将结合我自己对人工智能的理解,从历史,现在和未来三个角度对其进行分析。\par
老师的讲课方式是我一直以来所认同和期待的,很多历史上的伟大发明和创造,老师都能结合当时的时代情况,从一个不同的视角做出分析,得出其伟大的原因。正如中国的历史学家说,“以史为鉴,可知兴替”,从我们现在的视角去看过去的发明和创造,更加能指导我们现在的发明和创造。这其实也和人工智能的学习方式所类似,我们以历史上这些推动社会发展的创造为训练数据,让机器进行学习, 再以现在的社会现状等各方面因素作为输入,便可得到在未来,什么会是又一个改变世界的创举。\par
而我一直认为,人工智能正是这个时代的范式,而我本人也对这方面颇有兴趣,在本文中,我将结合我自己对人工智能的理解,从历史,现在和未来三个角度对其进行分析。\par
老师的讲课方式是我一直以来所认同和期待的,很多历史上的伟大发明和创造,老师都能结合当时的时代情况,从一个不同的视角做出分析,得出其伟大的原因。正如中国的历史学家说,“以史为鉴,可知兴替”,从我们现在的视角去看过去的发明和创造,更加能指导我们现在的发明和创造。这其实也和人工智能的学习方式所类似,我们以历史上这些推动社会发展的创造为训练数据,让机器进行学习, 再以现在的社会现状等各方面因素作为输入,便可得到在未来,什么会是又一个改变世界的创举。\par
而我一直认为,人工智能正是这个时代的范式,而我本人也对这方面颇有兴趣,在本文中,我将结合我自己对人工智能的理解,从历史,现在和未来三个角度对其进行分析。\par

\subsection{人工智能的提出}
其实人工智能这个概念在1950年左右就已经被提出,但一直没能得到较好的发展。在其发展的早期,比较重要的一个事件就是图灵所写的人工智能的第一篇论文——《机器能思考吗》的发表。图灵在这篇论文中,就机器是否具有思维这个问题进行了深入的探讨。其中他对一个观点的反对让我印象深刻,斐逊教授在1949年的一次演讲中曾说过,若要承认机器能像人一样思考,除非机器能够受感情影响而写出一首十四行诗,并意识到自己写了。而任何机器都感觉不到喜悦或是沮丧。图灵对这个观点的反驳是对我启发最大的一段话,“你若要肯定一台机器能思维,唯一的途径就是成为那台机器并去感受自己的思维活动”。一台机器,或者说人工智能能不能思考?我的回答是,肯定能思考,只不过不是以人类的方式来思考,也不是以我们能理解的方式来思考。举一个简单的例子, 我们不能理解动物的交流方式,难道我们能说动物没有交流吗?\par
在查阅了很多谈论人工智能发展的论文中,我发现这些论文中都忽略了一点,那就是我们为什么会需要发展人工智能?我从高中便开始了解一些人工智能的讯息,对这一问题也有一些自己的看法。我认为,我们需要人工智能的一个很重要的原因,那就是我们逐渐发现了人脑的不足。虽说人脑是大自然长期以来优胜劣汰的结果,它使得我们有区别与其他动物,然而,在实际中,人脑却有很多无法解决的问题,比如说数据的计算,这貌似是人脑天生的一个劣势,再比如说,在面对一些复杂的任务,比如开发一个产品时,往往需要多人协作,花费较长时间完成。这些都是人脑不断暴露出来的一些不足,而电脑,以其每秒上亿次的计算能力彻底惊艳了人们,便有人想到利用电脑的这个特点来弥补人脑的这些不足。于是,科学家便提出了各种赋予计算机思维的方法,便促成了现在人工智能的发展。\par
其实人工智能这个概念在1950年左右就已经被提出,但一直没能得到较好的发展。在其发展的早期,比较重要的一个事件就是图灵所写的人工智能的第一篇论文——《机器能思考吗》的发表。图灵在这篇论文中,就机器是否具有思维这个问题进行了深入的探讨。其中他对一个观点的反对让我印象深刻,斐逊教授在1949年的一次演讲中曾说过,若要承认机器能像人一样思考,除非机器能够受感情影响而写出一首十四行诗,并意识到自己写了。而任何机器都感觉不到喜悦或是沮丧。图灵对这个观点的反驳是对我启发最大的一段话,“你若要肯定一台机器能思维,唯一的途径就是成为那台机器并去感受自己的思维活动”。一台机器,或者说人工智能能不能思考?我的回答是,肯定能思考,只不过不是以人类的方式来思考,也不是以我们能理解的方式来思考。举一个简单的例子, 我们不能理解动物的交流方式,难道我们能说动物没有交流吗?\par
在查阅了很多谈论人工智能发展的论文中,我发现这些论文中都忽略了一点,那就是我们为什么会需要发展人工智能?我从高中便开始了解一些人工智能的讯息,对这一问题也有一些自己的看法。我认为,我们需要人工智能的一个很重要的原因,那就是我们逐渐发现了人脑的不足。虽说人脑是大自然长期以来优胜劣汰的结果,它使得我们有区别与其他动物,然而,在实际中,人脑却有很多无法解决的问题,比如说数据的计算,这貌似是人脑天生的一个劣势,再比如说,在面对一些复杂的任务,比如开发一个产品时,往往需要多人协作,花费较长时间完成。这些都是人脑不断暴露出来的一些不足,而电脑,以其每秒上亿次的计算能力彻底惊艳了人们,便有人想到利用电脑的这个特点来弥补人脑的这些不足。于是,科学家便提出了各种赋予计算机思维的方法,便促成了现在人工智能的发展。\par
其实人工智能这个概念在1950年左右就已经被提出,但一直没能得到较好的发展。在其发展的早期,比较重要的一个事件就是图灵所写的人工智能的第一篇论文——《机器能思考吗》的发表。图灵在这篇论文中,就机器是否具有思维这个问题进行了深入的探讨。其中他对一个观点的反对让我印象深刻,斐逊教授在1949年的一次演讲中曾说过,若要承认机器能像人一样思考,除非机器能够受感情影响而写出一首十四行诗,并意识到自己写了。而任何机器都感觉不到喜悦或是沮丧。图灵对这个观点的反驳是对我启发最大的一段话,“你若要肯定一台机器能思维,唯一的途径就是成为那台机器并去感受自己的思维活动”。一台机器,或者说人工智能能不能思考?我的回答是,肯定能思考,只不过不是以人类的方式来思考,也不是以我们能理解的方式来思考。举一个简单的例子, 我们不能理解动物的交流方式,难道我们能说动物没有交流吗?\par
在查阅了很多谈论人工智能发展的论文中,我发现这些论文中都忽略了一点,那就是我们为什么会需要发展人工智能?我从高中便开始了解一些人工智能的讯息,对这一问题也有一些自己的看法。我认为,我们需要人工智能的一个很重要的原因,那就是我们逐渐发现了人脑的不足。虽说人脑是大自然长期以来优胜劣汰的结果,它使得我们有区别与其他动物,然而,在实际中,人脑却有很多无法解决的问题,比如说数据的计算,这貌似是人脑天生的一个劣势,再比如说,在面对一些复杂的任务,比如开发一个产品时,往往需要多人协作,花费较长时间完成。这些都是人脑不断暴露出来的一些不足,而电脑,以其每秒上亿次的计算能力彻底惊艳了人们,便有人想到利用电脑的这个特点来弥补人脑的这些不足。于是,科学家便提出了各种赋予计算机思维的方法,便促成了现在人工智能的发展。\par
其实人工智能这个概念在1950年左右就已经被提出,但一直没能得到较好的发展。在其发展的早期,比较重要的一个事件就是图灵所写的人工智能的第一篇论文——《机器能思考吗》的发表。图灵在这篇论文中,就机器是否具有思维这个问题进行了深入的探讨。其中他对一个观点的反对让我印象深刻,斐逊教授在1949年的一次演讲中曾说过,若要承认机器能像人一样思考,除非机器能够受感情影响而写出一首十四行诗,并意识到自己写了。而任何机器都感觉不到喜悦或是沮丧。图灵对这个观点的反驳是对我启发最大的一段话,“你若要肯定一台机器能思维,唯一的途径就是成为那台机器并去感受自己的思维活动”。一台机器,或者说人工智能能不能思考?我的回答是,肯定能思考,只不过不是以人类的方式来思考,也不是以我们能理解的方式来思考。举一个简单的例子, 我们不能理解动物的交流方式,难道我们能说动物没有交流吗?\par
在查阅了很多谈论人工智能发展的论文中,我发现这些论文中都忽略了一点,那就是我们为什么会需要发展人工智能?我从高中便开始了解一些人工智能的讯息,对这一问题也有一些自己的看法。我认为,我们需要人工智能的一个很重要的原因,那就是我们逐渐发现了人脑的不足。虽说人脑是大自然长期以来优胜劣汰的结果,它使得我们有区别与其他动物,然而,在实际中,人脑却有很多无法解决的问题,比如说数据的计算,这貌似是人脑天生的一个劣势,再比如说,在面对一些复杂的任务,比如开发一个产品时,往往需要多人协作,花费较长时间完成。这些都是人脑不断暴露出来的一些不足,而电脑,以其每秒上亿次的计算能力彻底惊艳了人们,便有人想到利用电脑的这个特点来弥补人脑的这些不足。于是,科学家便提出了各种赋予计算机思维的方法,便促成了现在人工智能的发展。\par
其实人工智能这个概念在1950年左右就已经被提出,但一直没能得到较好的发展。在其发展的早期,比较重要的一个事件就是图灵所写的人工智能的第一篇论文——《机器能思考吗》的发表。图灵在这篇论文中,就机器是否具有思维这个问题进行了深入的探讨。其中他对一个观点的反对让我印象深刻,斐逊教授在1949年的一次演讲中曾说过,若要承认机器能像人一样思考,除非机器能够受感情影响而写出一首十四行诗,并意识到自己写了。而任何机器都感觉不到喜悦或是沮丧。图灵对这个观点的反驳是对我启发最大的一段话,“你若要肯定一台机器能思维,唯一的途径就是成为那台机器并去感受自己的思维活动”。一台机器,或者说人工智能能不能思考?我的回答是,肯定能思考,只不过不是以人类的方式来思考,也不是以我们能理解的方式来思考。举一个简单的例子, 我们不能理解动物的交流方式,难道我们能说动物没有交流吗?\par
在查阅了很多谈论人工智能发展的论文中,我发现这些论文中都忽略了一点,那就是我们为什么会需要发展人工智能?我从高中便开始了解一些人工智能的讯息,对这一问题也有一些自己的看法。我认为,我们需要人工智能的一个很重要的原因,那就是我们逐渐发现了人脑的不足。虽说人脑是大自然长期以来优胜劣汰的结果,它使得我们有区别与其他动物,然而,在实际中,人脑却有很多无法解决的问题,比如说数据的计算,这貌似是人脑天生的一个劣势,再比如说,在面对一些复杂的任务,比如开发一个产品时,往往需要多人协作,花费较长时间完成。这些都是人脑不断暴露出来的一些不足,而电脑,以其每秒上亿次的计算能力彻底惊艳了人们,便有人想到利用电脑的这个特点来弥补人脑的这些不足。于是,科学家便提出了各种赋予计算机思维的方法,便促成了现在人工智能的发展。\par

\subsubsection{人工智能的的现状}
人工智能自被提出到现在,经历了几次低谷,但现在,得益于计算机计算能力的提升和数学家在这方面一些更加优秀的算法的提出,人工智能进入了前所未有的繁荣期。光回顾近几年人工智能的发展,就有好几个值得我们关注的里程碑的时间,无人车开始进入大众视野、语音识别的准确率达$97\%$、AlphaGo战胜人类顶尖棋手李世石和柯洁。我们甚至可以看到,很多人工智能的产品已经进入了我们大众的视野,比如说,语音输入成为输入的新潮流,各网站会根据你的习惯为你推送你喜欢的东西……\par
然而,仔细思考便会发现,大部分的人工智能都还处于试验阶段,这些所谓的人工智能只是应用于非常小的一部分功能。我曾在几星期前有幸去听了郑南宁院士关于无人车的一场讲座,郑教授在讲座中提到,目前无人车在一些较为规则的路况下基本可以保证正常行驶,但当遇到较复杂的交通状况,比如有各方向的行人和来车的十字路口,无人车就难以做出正确的判断,其中的主要原因包括无法准确识别道路中各交通参与者、无法准确预测交通参与者的运动方向。而近期发生的首件无人车致人死亡事件,更是说明了人工智能并不像某些科技公司所宣扬的那样马上就要到来,而是还有很长的路要走。\par
另一个值得我们关注的便是对与人工智能发展的担忧,害怕机器会超过人类,甚至战胜人类。其实,这样的观点在历史上并不是第一次了,回顾历史上那些改变世界的发明创造,有很多在当时都被认为是奇怪、危险的。飞机、电力、照相机……,这些东西都曾让人们担忧,而时间终将证明其伟大。而人们对于人工智能的担忧也并不是无根据的,仍然以无人车举例,从一开始的无人车撞车,到现在的无人车致人死亡,人们越来越对无人车的安全性感到怀疑。在郑院士的讲座中,有同学问,还要多久才能坐上一辆“交大出产”的无人车,郑院士客观地回答道,从科研者的角度来说,至少还得十多年。可见,无人驾驶技术还没有完全成熟,现在的一些公司将其市场化,才导致了现在这些无人车事故的出现。而假以时日,人工智能技术必将真正走向成熟,有效地避免这些事件的发生。\par
人工智能自被提出到现在,经历了几次低谷,但现在,得益于计算机计算能力的提升和数学家在这方面一些更加优秀的算法的提出,人工智能进入了前所未有的繁荣期。光回顾近几年人工智能的发展,就有好几个值得我们关注的里程碑的时间,无人车开始进入大众视野、语音识别的准确率达$97\%$、AlphaGo战胜人类顶尖棋手李世石和柯洁。我们甚至可以看到,很多人工智能的产品已经进入了我们大众的视野,比如说,语音输入成为输入的新潮流,各网站会根据你的习惯为你推送你喜欢的东西……\par
然而,仔细思考便会发现,大部分的人工智能都还处于试验阶段,这些所谓的人工智能只是应用于非常小的一部分功能。我曾在几星期前有幸去听了郑南宁院士关于无人车的一场讲座,郑教授在讲座中提到,目前无人车在一些较为规则的路况下基本可以保证正常行驶,但当遇到较复杂的交通状况,比如有各方向的行人和来车的十字路口,无人车就难以做出正确的判断,其中的主要原因包括无法准确识别道路中各交通参与者、无法准确预测交通参与者的运动方向。而近期发生的首件无人车致人死亡事件,更是说明了人工智能并不像某些科技公司所宣扬的那样马上就要到来,而是还有很长的路要走。\par
另一个值得我们关注的便是对与人工智能发展的担忧,害怕机器会超过人类,甚至战胜人类。其实,这样的观点在历史上并不是第一次了,回顾历史上那些改变世界的发明创造,有很多在当时都被认为是奇怪、危险的。飞机、电力、照相机……,这些东西都曾让人们担忧,而时间终将证明其伟大。而人们对于人工智能的担忧也并不是无根据的,仍然以无人车举例,从一开始的无人车撞车,到现在的无人车致人死亡,人们越来越对无人车的安全性感到怀疑。在郑院士的讲座中,有同学问,还要多久才能坐上一辆“交大出产”的无人车,郑院士客观地回答道,从科研者的角度来说,至少还得十多年。可见,无人驾驶技术还没有完全成熟,现在的一些公司将其市场化,才导致了现在这些无人车事故的出现。而假以时日,人工智能技术必将真正走向成熟,有效地避免这些事件的发生。\par
人工智能自被提出到现在,经历了几次低谷,但现在,得益于计算机计算能力的提升和数学家在这方面一些更加优秀的算法的提出,人工智能进入了前所未有的繁荣期。光回顾近几年人工智能的发展,就有好几个值得我们关注的里程碑的时间,无人车开始进入大众视野、语音识别的准确率达$97\%$、AlphaGo战胜人类顶尖棋手李世石和柯洁。我们甚至可以看到,很多人工智能的产品已经进入了我们大众的视野,比如说,语音输入成为输入的新潮流,各网站会根据你的习惯为你推送你喜欢的东西……\par
然而,仔细思考便会发现,大部分的人工智能都还处于试验阶段,这些所谓的人工智能只是应用于非常小的一部分功能。我曾在几星期前有幸去听了郑南宁院士关于无人车的一场讲座,郑教授在讲座中提到,目前无人车在一些较为规则的路况下基本可以保证正常行驶,但当遇到较复杂的交通状况,比如有各方向的行人和来车的十字路口,无人车就难以做出正确的判断,其中的主要原因包括无法准确识别道路中各交通参与者、无法准确预测交通参与者的运动方向。而近期发生的首件无人车致人死亡事件,更是说明了人工智能并不像某些科技公司所宣扬的那样马上就要到来,而是还有很长的路要走。\par
另一个值得我们关注的便是对与人工智能发展的担忧,害怕机器会超过人类,甚至战胜人类。其实,这样的观点在历史上并不是第一次了,回顾历史上那些改变世界的发明创造,有很多在当时都被认为是奇怪、危险的。飞机、电力、照相机……,这些东西都曾让人们担忧,而时间终将证明其伟大。而人们对于人工智能的担忧也并不是无根据的,仍然以无人车举例,从一开始的无人车撞车,到现在的无人车致人死亡,人们越来越对无人车的安全性感到怀疑。在郑院士的讲座中,有同学问,还要多久才能坐上一辆“交大出产”的无人车,郑院士客观地回答道,从科研者的角度来说,至少还得十多年。可见,无人驾驶技术还没有完全成熟,现在的一些公司将其市场化,才导致了现在这些无人车事故的出现。而假以时日,人工智能技术必将真正走向成熟,有效地避免这些事件的发生。\par
人工智能自被提出到现在,经历了几次低谷,但现在,得益于计算机计算能力的提升和数学家在这方面一些更加优秀的算法的提出,人工智能进入了前所未有的繁荣期。光回顾近几年人工智能的发展,就有好几个值得我们关注的里程碑的时间,无人车开始进入大众视野、语音识别的准确率达$97\%$、AlphaGo战胜人类顶尖棋手李世石和柯洁。我们甚至可以看到,很多人工智能的产品已经进入了我们大众的视野,比如说,语音输入成为输入的新潮流,各网站会根据你的习惯为你推送你喜欢的东西……\par
然而,仔细思考便会发现,大部分的人工智能都还处于试验阶段,这些所谓的人工智能只是应用于非常小的一部分功能。我曾在几星期前有幸去听了郑南宁院士关于无人车的一场讲座,郑教授在讲座中提到,目前无人车在一些较为规则的路况下基本可以保证正常行驶,但当遇到较复杂的交通状况,比如有各方向的行人和来车的十字路口,无人车就难以做出正确的判断,其中的主要原因包括无法准确识别道路中各交通参与者、无法准确预测交通参与者的运动方向。而近期发生的首件无人车致人死亡事件,更是说明了人工智能并不像某些科技公司所宣扬的那样马上就要到来,而是还有很长的路要走。\par
另一个值得我们关注的便是对与人工智能发展的担忧,害怕机器会超过人类,甚至战胜人类。其实,这样的观点在历史上并不是第一次了,回顾历史上那些改变世界的发明创造,有很多在当时都被认为是奇怪、危险的。飞机、电力、照相机……,这些东西都曾让人们担忧,而时间终将证明其伟大。而人们对于人工智能的担忧也并不是无根据的,仍然以无人车举例,从一开始的无人车撞车,到现在的无人车致人死亡,人们越来越对无人车的安全性感到怀疑。在郑院士的讲座中,有同学问,还要多久才能坐上一辆“交大出产”的无人车,郑院士客观地回答道,从科研者的角度来说,至少还得十多年。可见,无人驾驶技术还没有完全成熟,现在的一些公司将其市场化,才导致了现在这些无人车事故的出现。而假以时日,人工智能技术必将真正走向成熟,有效地避免这些事件的发生。\par
人工智能自被提出到现在,经历了几次低谷,但现在,得益于计算机计算能力的提升和数学家在这方面一些更加优秀的算法的提出,人工智能进入了前所未有的繁荣期。光回顾近几年人工智能的发展,就有好几个值得我们关注的里程碑的时间,无人车开始进入大众视野、语音识别的准确率达$97\%$、AlphaGo战胜人类顶尖棋手李世石和柯洁。我们甚至可以看到,很多人工智能的产品已经进入了我们大众的视野,比如说,语音输入成为输入的新潮流,各网站会根据你的习惯为你推送你喜欢的东西……\par
然而,仔细思考便会发现,大部分的人工智能都还处于试验阶段,这些所谓的人工智能只是应用于非常小的一部分功能。我曾在几星期前有幸去听了郑南宁院士关于无人车的一场讲座,郑教授在讲座中提到,目前无人车在一些较为规则的路况下基本可以保证正常行驶,但当遇到较复杂的交通状况,比如有各方向的行人和来车的十字路口,无人车就难以做出正确的判断,其中的主要原因包括无法准确识别道路中各交通参与者、无法准确预测交通参与者的运动方向。而近期发生的首件无人车致人死亡事件,更是说明了人工智能并不像某些科技公司所宣扬的那样马上就要到来,而是还有很长的路要走。\par
另一个值得我们关注的便是对与人工智能发展的担忧,害怕机器会超过人类,甚至战胜人类。其实,这样的观点在历史上并不是第一次了,回顾历史上那些改变世界的发明创造,有很多在当时都被认为是奇怪、危险的。飞机、电力、照相机……,这些东西都曾让人们担忧,而时间终将证明其伟大。而人们对于人工智能的担忧也并不是无根据的,仍然以无人车举例,从一开始的无人车撞车,到现在的无人车致人死亡,人们越来越对无人车的安全性感到怀疑。在郑院士的讲座中,有同学问,还要多久才能坐上一辆“交大出产”的无人车,郑院士客观地回答道,从科研者的角度来说,至少还得十多年。可见,无人驾驶技术还没有完全成熟,现在的一些公司将其市场化,才导致了现在这些无人车事故的出现。而假以时日,人工智能技术必将真正走向成熟,有效地避免这些事件的发生。\par

\subsubsection{人工智能的未来}
最早让我关注到人工智能的,其实是电影《钢铁侠》中主角的智能助理——贾维斯。虽说电影是虚构的,但我从那时到现在,一直认为电影中的贾维斯,就应该是人工智能的未来,是人工智能最终的样子。人工智能不应该只是像Siri、Cortana那样只能聊天的简单软件,它应该像贾维斯那样,能帮助人类完成几乎所有事情。对于个人而言,人工智能将会是一个全能的助手,人类只需对这个系统做初始的编程,而人工智能会自己去学习新的技能,不再需要人类针对其某一特定功能进行长时间的编程。换句话说,未来的人工智能应该是一种能够自己进化的系统,而凭借计算机非凡的计算能力,或许只要几分钟,就能完成人类长达几亿年的进化过程。然而,也有人可能会担心人工智能这种自主的进化或许会威胁到人类自身。其实,我认为,人工智能的进化过程再很大程度上是可以由我们自己来控制的,就像达尔文的物种进化理论,我们完全可以通过控制人工智能进化的环境,让最适合人类的人工智能留下,将那些有害的扼杀再摇篮里。\par
然而,我觉得人工智能施展身手的最大领域还不是民用。中国古代哲学家王阳明有“格物致知”的一套学说,但最终以自己格竹子七天未果而放弃。其实西方科学家也有类似的言论,一位西方的科学家曾经说过,如果能知道宇宙开始状态的所有粒子的状态,那么便能够预测未来每一秒的情况。这些言论在当时看起来很荒谬,很大一部分原因是他们只是理论上正确的,而实际上,人类的智力水平远远达不到这两种理论的要求。不过,现在有了人工智能,我们或许应该重新审视一下这些个理论,凭借人工智能的计算能力,格物致知,预测未来或许都不再将是传说。\par
人工智能的潜力是无限的,如果能够正确使用,它绝对将会是颠覆21实际的一股力量。\par
最早让我关注到人工智能的,其实是电影《钢铁侠》中主角的智能助理——贾维斯。虽说电影是虚构的,但我从那时到现在,一直认为电影中的贾维斯,就应该是人工智能的未来,是人工智能最终的样子。人工智能不应该只是像Siri、Cortana那样只能聊天的简单软件,它应该像贾维斯那样,能帮助人类完成几乎所有事情。对于个人而言,人工智能将会是一个全能的助手,人类只需对这个系统做初始的编程,而人工智能会自己去学习新的技能,不再需要人类针对其某一特定功能进行长时间的编程。换句话说,未来的人工智能应该是一种能够自己进化的系统,而凭借计算机非凡的计算能力,或许只要几分钟,就能完成人类长达几亿年的进化过程。然而,也有人可能会担心人工智能这种自主的进化或许会威胁到人类自身。其实,我认为,人工智能的进化过程再很大程度上是可以由我们自己来控制的,就像达尔文的物种进化理论,我们完全可以通过控制人工智能进化的环境,让最适合人类的人工智能留下,将那些有害的扼杀再摇篮里。\par
然而,我觉得人工智能施展身手的最大领域还不是民用。中国古代哲学家王阳明有“格物致知”的一套学说,但最终以自己格竹子七天未果而放弃。其实西方科学家也有类似的言论,一位西方的科学家曾经说过,如果能知道宇宙开始状态的所有粒子的状态,那么便能够预测未来每一秒的情况。这些言论在当时看起来很荒谬,很大一部分原因是他们只是理论上正确的,而实际上,人类的智力水平远远达不到这两种理论的要求。不过,现在有了人工智能,我们或许应该重新审视一下这些个理论,凭借人工智能的计算能力,格物致知,预测未来或许都不再将是传说。\par
人工智能的潜力是无限的,如果能够正确使用,它绝对将会是颠覆21实际的一股力量。\par
最早让我关注到人工智能的,其实是电影《钢铁侠》中主角的智能助理——贾维斯。虽说电影是虚构的,但我从那时到现在,一直认为电影中的贾维斯,就应该是人工智能的未来,是人工智能最终的样子。人工智能不应该只是像Siri、Cortana那样只能聊天的简单软件,它应该像贾维斯那样,能帮助人类完成几乎所有事情。对于个人而言,人工智能将会是一个全能的助手,人类只需对这个系统做初始的编程,而人工智能会自己去学习新的技能,不再需要人类针对其某一特定功能进行长时间的编程。换句话说,未来的人工智能应该是一种能够自己进化的系统,而凭借计算机非凡的计算能力,或许只要几分钟,就能完成人类长达几亿年的进化过程。然而,也有人可能会担心人工智能这种自主的进化或许会威胁到人类自身。其实,我认为,人工智能的进化过程再很大程度上是可以由我们自己来控制的,就像达尔文的物种进化理论,我们完全可以通过控制人工智能进化的环境,让最适合人类的人工智能留下,将那些有害的扼杀再摇篮里。\par
然而,我觉得人工智能施展身手的最大领域还不是民用。中国古代哲学家王阳明有“格物致知”的一套学说,但最终以自己格竹子七天未果而放弃。其实西方科学家也有类似的言论,一位西方的科学家曾经说过,如果能知道宇宙开始状态的所有粒子的状态,那么便能够预测未来每一秒的情况。这些言论在当时看起来很荒谬,很大一部分原因是他们只是理论上正确的,而实际上,人类的智力水平远远达不到这两种理论的要求。不过,现在有了人工智能,我们或许应该重新审视一下这些个理论,凭借人工智能的计算能力,格物致知,预测未来或许都不再将是传说。\par
人工智能的潜力是无限的,如果能够正确使用,它绝对将会是颠覆21实际的一股力量。\par
最早让我关注到人工智能的,其实是电影《钢铁侠》中主角的智能助理——贾维斯。虽说电影是虚构的,但我从那时到现在,一直认为电影中的贾维斯,就应该是人工智能的未来,是人工智能最终的样子。人工智能不应该只是像Siri、Cortana那样只能聊天的简单软件,它应该像贾维斯那样,能帮助人类完成几乎所有事情。对于个人而言,人工智能将会是一个全能的助手,人类只需对这个系统做初始的编程,而人工智能会自己去学习新的技能,不再需要人类针对其某一特定功能进行长时间的编程。换句话说,未来的人工智能应该是一种能够自己进化的系统,而凭借计算机非凡的计算能力,或许只要几分钟,就能完成人类长达几亿年的进化过程。然而,也有人可能会担心人工智能这种自主的进化或许会威胁到人类自身。其实,我认为,人工智能的进化过程再很大程度上是可以由我们自己来控制的,就像达尔文的物种进化理论,我们完全可以通过控制人工智能进化的环境,让最适合人类的人工智能留下,将那些有害的扼杀再摇篮里。\par
然而,我觉得人工智能施展身手的最大领域还不是民用。中国古代哲学家王阳明有“格物致知”的一套学说,但最终以自己格竹子七天未果而放弃。其实西方科学家也有类似的言论,一位西方的科学家曾经说过,如果能知道宇宙开始状态的所有粒子的状态,那么便能够预测未来每一秒的情况。这些言论在当时看起来很荒谬,很大一部分原因是他们只是理论上正确的,而实际上,人类的智力水平远远达不到这两种理论的要求。不过,现在有了人工智能,我们或许应该重新审视一下这些个理论,凭借人工智能的计算能力,格物致知,预测未来或许都不再将是传说。\par
人工智能的潜力是无限的,如果能够正确使用,它绝对将会是颠覆21实际的一股力量。\par
最早让我关注到人工智能的,其实是电影《钢铁侠》中主角的智能助理——贾维斯。虽说电影是虚构的,但我从那时到现在,一直认为电影中的贾维斯,就应该是人工智能的未来,是人工智能最终的样子。人工智能不应该只是像Siri、Cortana那样只能聊天的简单软件,它应该像贾维斯那样,能帮助人类完成几乎所有事情。对于个人而言,人工智能将会是一个全能的助手,人类只需对这个系统做初始的编程,而人工智能会自己去学习新的技能,不再需要人类针对其某一特定功能进行长时间的编程。换句话说,未来的人工智能应该是一种能够自己进化的系统,而凭借计算机非凡的计算能力,或许只要几分钟,就能完成人类长达几亿年的进化过程。然而,也有人可能会担心人工智能这种自主的进化或许会威胁到人类自身。其实,我认为,人工智能的进化过程再很大程度上是可以由我们自己来控制的,就像达尔文的物种进化理论,我们完全可以通过控制人工智能进化的环境,让最适合人类的人工智能留下,将那些有害的扼杀再摇篮里。\par
然而,我觉得人工智能施展身手的最大领域还不是民用。中国古代哲学家王阳明有“格物致知”的一套学说,但最终以自己格竹子七天未果而放弃。其实西方科学家也有类似的言论,一位西方的科学家曾经说过,如果能知道宇宙开始状态的所有粒子的状态,那么便能够预测未来每一秒的情况。这些言论在当时看起来很荒谬,很大一部分原因是他们只是理论上正确的,而实际上,人类的智力水平远远达不到这两种理论的要求。不过,现在有了人工智能,我们或许应该重新审视一下这些个理论,凭借人工智能的计算能力,格物致知,预测未来或许都不再将是传说。\par
人工智能的潜力是无限的,如果能够正确使用,它绝对将会是颠覆21实际的一股力量。\par
最早让我关注到人工智能的,其实是电影《钢铁侠》中主角的智能助理——贾维斯。虽说电影是虚构的,但我从那时到现在,一直认为电影中的贾维斯,就应该是人工智能的未来,是人工智能最终的样子。人工智能不应该只是像Siri、Cortana那样只能聊天的简单软件,它应该像贾维斯那样,能帮助人类完成几乎所有事情。对于个人而言,人工智能将会是一个全能的助手,人类只需对这个系统做初始的编程,而人工智能会自己去学习新的技能,不再需要人类针对其某一特定功能进行长时间的编程。换句话说,未来的人工智能应该是一种能够自己进化的系统,而凭借计算机非凡的计算能力,或许只要几分钟,就能完成人类长达几亿年的进化过程。然而,也有人可能会担心人工智能这种自主的进化或许会威胁到人类自身。其实,我认为,人工智能的进化过程再很大程度上是可以由我们自己来控制的,就像达尔文的物种进化理论,我们完全可以通过控制人工智能进化的环境,让最适合人类的人工智能留下,将那些有害的扼杀再摇篮里。\par
然而,我觉得人工智能施展身手的最大领域还不是民用。中国古代哲学家王阳明有“格物致知”的一套学说,但最终以自己格竹子七天未果而放弃。其实西方科学家也有类似的言论,一位西方的科学家曾经说过,如果能知道宇宙开始状态的所有粒子的状态,那么便能够预测未来每一秒的情况。这些言论在当时看起来很荒谬,很大一部分原因是他们只是理论上正确的,而实际上,人类的智力水平远远达不到这两种理论的要求。不过,现在有了人工智能,我们或许应该重新审视一下这些个理论,凭借人工智能的计算能力,格物致知,预测未来或许都不再将是传说。\par
人工智能的潜力是无限的,如果能够正确使用,它绝对将会是颠覆21实际的一股力量。\par
\end{document} 